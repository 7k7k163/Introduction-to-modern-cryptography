% !TEX root = ../../现代密码学简介.tex
\chapter{对称密码}
\section{什么是对称密码}
通过上一章我们知道,一个密码体制由明文空间、密文空间、密钥空间、加密算法、解密算法构成。如果两个人要进行加密通信,那么发送者利用密钥,使用加密算法加密明文,生成密文,将密文传递给接收者。接收者利用密钥,使用解密算法解密密文,生成明文。这就是加密通信的基本过程。那么所谓的对称密码,就是在一次加密通信的过程中,发送者和接收者使用的密钥是同一个密钥。
\section{对称密码要研究什么}
从我们上面叙述的过程来看,对称密码主要需要研究两个部分:用什么方式加密、解密,以及如何让加密和解密者共享同一个密钥。\par
所谓用什么方式加密、解密,主流方法为使用流密码或分组密码。而密钥共享的方式,则有许多种。本章主要的内容就是阐述这两个方面。
\section{对称密码的加密、解密方式}
从上一章我们可以知道,最安全的密码,是一次一密。即:\par
对于$n$位二进制串明文$m_1m_2\cdots m_n$, 密钥为$n$位二进制串$k_1k_2\cdots k_n$,其每一位都是$0$或$1$的随机数。密文为$n$位二进制串$c_1c_2\cdots c_n$, 加密方式为明文与密钥的逐比特异或,解密方式为密文与密钥的逐比特异或。\par
同时我们知道,一次一密虽然是完全安全的,但其缺陷也很明显。首先,既然密钥的长度和明文一样长,并且在加密通信前双方都要知道密钥,那不如双方直接共享明文;其次,同一个随机生成的密钥只能使用于一次加密通信,否则安全性就会降低。