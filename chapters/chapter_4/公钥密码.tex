% !TEX root = 现代密码学简介
\chapter{公钥密码}
\section{简介}
之前我们讲到,最安全的密码体系是一次一密。但是,由于其需要用安全信道传输的密钥长度至少于明文一样长,并且一个密钥只能用一次,因此,与其将密钥传输,不如将明文传输。所以,一次一密的缺陷也十分明显。为了解决这一问题,之前我们采取的方法是使用各种手法,包括流密码以及分组密码等,来缩短密钥的长度。而公钥密码则提供了另一种不同的思路:减弱对安全信道的需求。其基本思想为:
\begin{enumerate}
	\item 由解密方生成一对密钥,称为公钥(记作$SK$)和私钥(记作$PK$)。
	\item 解密方将公钥传送给加密方(不需要通过安全信道)。
	\item 加密方利用公钥加密明文,传递给解密方。
	\item 解密方利用私钥解密密文。
\end{enumerate}

要在实践中实现这个思想,我们得满足:
\begin{itemize}
	\item 生成公钥、密钥对的算法较容易
	\item 用公钥加密明文的算法较容易
	\item 用密钥解密密文的算法较容易
	\item 由公钥不能(或者很难)得到对应的密钥
	\item 由密文和公钥不能(或者很难)得到对应的明文
	\item 公钥、私钥可交换。即:
	\begin{equation}
	\D{PK}{\E{SK}{m}} = \D{SK}{\E{PK}{m}}
	\end{equation}
\end{itemize}

此外,公钥密码体系还有一个重要的特点是,其安全性基于数论知识,而非类似于分组密码的代换与替换。并且,在之后的算法细节中我们会了解到,公钥密码体系的算法比对称密码体系的算法要慢许多。同时,由于并非一次一密,因此其安全性从本质上说并不比对称密码体系高。\par
公钥密码体系也称为非对称密码体系,因为用于加密的公钥不能解密其密文。\par
同时,公钥密码除了解决了密钥分配问题,即对安全信道的需求问题,其还解决了另外一个问题:数字签名问题。即,在以往的对称加密体系中,我们无法得知收到的密文是否来自于我们选择的加密方。