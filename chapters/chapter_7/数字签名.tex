% !TEX root = ../../现代密码学简介.tex
\chapter{数字签名}
\section{简介}
通过上一章讲到的MAC以及认证加密,我们有效地解决了消息认证的前两个问题:完整性与真实性。接下来,我们介绍一种强大的工具——数字签名。这种工具能同时解决完整性、真实性与不可否认性这三个消息认证的问题。下面这张著名的表格归纳了消息认证的三个手段与其特点:
\begin{table}[H]
    \centering
    \begin{tabular}{c|c|c|c}\hline
        &哈希函数&MAC&数字签名\\\hline
        完整性&是&是&是\\\hline
        真实性&否&是&是\\\hline
        不可否认性&否&否&是\\\hline
        密钥类型&无密钥&对称密钥&非对称密钥\\\hline
    \end{tabular}
\end{table}

数字签名需要第三方来进行验证。那么,我们之前了解到的工具中有什么东西是第三方知道的呢?我们最先想到的就是公钥密码体系中发送者的公钥。此外,也有一些单独设计的数字签名算法。由于涉及到了密钥,因此,数字签名总共有三个部分:
\begin{itemize}
    \item 密钥生成算法Gen\par
    返回用于签名的密钥(使用公钥密码时返回公-私钥对$\pth{PK, SK}$)
    \item 签名算法$\mathrm{Sig}\pth{M, SK}$\par
    返回签名$S$
    \item 验证签名算法$\mathrm{Ver}\pth{S, PK}$\par
    返回是否验证成功
\end{itemize}

此外,如果不讨论不可否认性,由于MAC使用的是对称密钥,数字签名使用的是公钥密码,那么数字签名实际上就是公钥密码领域中的消息验证码。
\section{数字签名的执行方法}
数字签名是为了一次性做到完整性、真实性和不可否认性的认证。用一句话概括,就是“这个消息确实是他说的”。那么,由谁来验证数字签名呢?最原始的方式是直接方式。
\subsection{直接方式(Direct digital signatures)}
即接收方直接利用验证签名算法和发送方的公钥验证发送方的签名。\par
这种方式依赖于发送方私钥的安全性。如果发送方私钥丢失,那么窃取者可以伪造签名;如果发送方声称私钥丢失,那么签名就失去了验证效力,消息得到了否认。这样的话,数字签名的真实性和不可否认性都不能得到保证。
\subsection{仲裁方式(Arbitrated digital signatures)}
对于仲裁方式,需要第三方$C$, 并且$C$应该收到发送方$A$和接收方$B$的高度信任,而且拥有$A$的公钥与加密用的密钥。$C$称为仲裁方(Arbiter)。发送方将消息及其签名先发给仲裁方。仲裁方验证了消息的完整性、真实性和不可否认性后,一般是加上时间戳,然后将消息发送给$B$.\par
下面介绍一些具体的数字签名算法。
\section{基于公钥密码的数字签名的基本步骤}
下面假设发送者是$A$, 接收者是$B$, 第三方是$C$, 要发送的消息为$M$, 采用的加密密钥为$K$, 哈希算法为$H$,用于加密消息的算法为$E$, 其对应的解密消息的算法为$D$, 此外还有一个公钥加密算法$\mathscr{E}$与其对应的解密算法$\mathscr{D}$. 这里要求用于数字签名的公钥密码需要满足一个性质:
\[\mathscr{D}_{PK}\pth{\mathscr{E}_{SK}{M}}=\mathscr{D}_{SK}\pth{\mathscr{E}_{PK}\pth{M}}\]

也就是说,公钥和私钥都可以用于加密或者解密。满足这一性质的公钥密码,最著名的就是RSA.
\subsection{密钥生成算法}
由发送者生成一对公钥-私钥对$\pth{PK, SK}$,同时向外公布公钥$PK$。\par
注意,发送者用于签名的公-私钥对不能用于加密。
\subsection{签名算法}
\[A\to B:\E{K}{M},\mathscr{E}_{SK}\pth{H\pth{M}}\]
\subsection{验证签名算法}
$B$收到$A$发送的两部分密文$C$与签名$S$. 先用双方共享的密钥$K$解密$C$得到$M'=\D{K}{C}$, 然后求其摘要$h'=H\pth{M'}$. 再用$A$公开的公钥$PK$解密$S$得到$h''=\mathscr{D}_{PK}\pth{S}$. 如果$h'=h''$, 则签名有效。
\subsection{讨论}
这里为什么需要哈希函数呢?我们如果不进行哈希,直接由$A$向$B$发送
\[A\to B:\E{K}{M},\mathscr{E}_{SK}\pth{M}\]

行不行呢?\par
由于我们之前提到的用于数字签名的公钥密码满足性质
\[\mathscr{D}_{PK}\pth{\mathscr{E}_{SK}{M}}=\mathscr{D}_{SK}\pth{\mathscr{E}_{PK}\pth{M}}\]

同时,$A$用于签名的公钥由于是广播给所有人的,所以,我们可以轻松地伪造一条由$A$签名的消息:\par
任意选择一个签名$S$, 计算$M=\mathscr{D}_{PK}\pth{S}$, 那么消息$\E{K}{M}, S$就会被成功验证为$A$签名的一条消息。\par
而如果我们增加了哈希,不仅可以验证消息的完整性,同时,也可以抵御这种攻击。
\section{数字签名算法DSA}
DSA是NIST推荐的数字签名算法。其算法如下:
\subsection{密钥生成算法}
\begin{enumerate}
    \item 选择整数对$\pth{L, N}$. 其可选择的取值有:
    \begin{itemize}
        \item $L=1024, N=160$
        \item $L=2048, N=224$
        \item $L=2048, N=256$
        \item $L=3072, N=256$
    \end{itemize}

    \item 生成一个长度为$N$的大素数$q$, 即$2^{N-1} < q < 2^N$.
    \item 生成一个长度为$L$的大素数$p$, 即$2^{L-1} < p < 2^L$. 且满足$q\mid p - 1$.
    \item 选择有限域$\GF\pth{p}$的一个$q$阶子群$G$, 选择$G$的一个生成元$g$, 满足$1<g<p$
    \item 生成一个随机数$x$满足$0<x<q$
    \item 计算$y=g^x\bmod{p}$
    \item 选择一种哈希算法$H$
\end{enumerate}

其中,$\pth{p, q, g}$称为全局公开钥,所有人都能获得;$x$为私钥,$y$为公钥。同时,双方还应知道使用的哈希算法$H$.\par
在实际操作过程中,由$p, q$生成$g$的方法如下:
\begin{enumerate}
    \item 计算$\ext e=\frac{p-1}{q}$
    \item\label{DSAgenH} 生成一个随机数$h$满足$1<h<p-1$.
    \item 计算$g=h^e\bmod{p}$
    \item 如果$g=1$, 则返回步骤\ref{DSAgenH}
\end{enumerate}
\subsection{签名算法}
假设明文为$M$, 全局公开钥为$\pth{p, q, g}$, 私钥为$x$.
\begin{enumerate}
    \item 生成一个随机数$k$, 满足$0<k<q$
    \item 计算$r=\pth{g^k\bmod{p}}\bmod{q}$
    \item 假设$k$模$q$的逆元为$k^{-1}$, 计算$s=\pth{k^{-1}\pth{H(M)+xr}}\bmod{q}$
\end{enumerate}

生成的签名为$\pth{r, s}$
\subsection{验证签名算法}
假设接收方解密得到的明文为$M'$, 签名为$\pth{r', s'}$, 全局公开钥为$\pth{p, q, g}$, 公钥为$y$.
\begin{enumerate}
    \item 假设$s'$模$q$的逆元为$\pth{s'}^{-1}$, 计算
    \begin{align*}
        w&=\pth{s'}^{-1}\bmod{q}\\
        u_1&=\pth{H\pth{M'}w}\bmod{q}\\
        u_2&=r'w\bmod{q}\\
        v&=\pth{\pth{g^{u_1}y^{u_2}\bmod{p}}\bmod{q}}
    \end{align*}
    \item 如果$v=r'$, 则签名有效;反之无效
\end{enumerate}
