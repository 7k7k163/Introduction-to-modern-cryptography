% !TEX root = ../../现代密码学简介.tex
\chapter{消息验证码}
\section{简介}
通过上一章的介绍,我们掌握了一个用于验证消息完整性的强大工具:哈希函数。通过比较发送者发送的摘要与接收者计算的摘要,接收者可以判断消息是否在传递过程中受到篡改。但是,这仅仅是第一步。在消息验证的过程中,还有一个非常重要的目的:在确定消息完整性的同时,确定消息发送自期望的发送者。因此,我们引入一个新的工具:消息验证码(Message Authentication Code, MAC).\par
在之前提到安全协议的时候,我们知道,密钥交换协议可以确保消息发送自期望的发送者。但是,密钥交换协议却不能确定消息的完整性。那么,如何同时做到确定消息的完整性,和确定消息发送自期望的发送者呢?\par
既然密钥交换协议可以确保消息发自期望的发送者,那么换个角度说,如果接收者能确定消息的发送者和自己使用的是相同的密钥,那么也就能确定消息的发送者是期望的发送者。因此,我们可以结合密钥和之前提到的哈希函数的基本概念,从而提出MAC算法$C_K(m)$的基本概念:
\begin{itemize}
	\item 该算法适用于任意长度的消息$m$
	\item 算法使用一个密钥$K$
	\item 该算法的输出$C_K(m)$是固定长的
	\item 由输入$m$和密钥$K$求得输出$C_K(m)$的过程较容易
	\item 由输出$C_K(m)$反求得输入$m$或密钥$K$的过程是不可行的,或者是难以计算的
	\item 给定$x$, 难以求得$y$, 使得$C_K(x)=C_K(y)$
	\item 难以求得任何一对$\pth{x, y}$, 使得$C_K(x)=C_K(y)$
\end{itemize}

我们可以看到,MAC算法的设计准则和哈希算法的设计准则几乎一致,只是MAC算法还需要一个密钥。此外,MAC算法的输出称为\textbf{标签}(tag).\par
发送方通过密钥$K$, 使用MAC算法对要传递的消息$m$进行处理,得到标签$t$, 将$\pth{m, t}$发送给接收者。接收方接收到$\pth{m', t}$后,通过与发送方共享的密钥$K$, 使用同样的MAC算法对接收到的消息$m'$进行处理,得到标签$t'$. 如果$t'=t$, 那么既说明了发送方发送的消息并没有得到篡改,也说明了发送方与自己使用的是同一个密钥,也就是消息的发送方是期待的发送方。\par
这就是MAC算法的基本思想。