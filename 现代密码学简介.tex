% !TEX program = xelatex
\documentclass[UTF8]{ctexrep}
\usepackage{xeCJK}
\setCJKmainfont[BoldFont=HiraginoSansGB-W3, ItalicFont=AdobeKaitiStd-Regular]{FZXSSJW--GB1-0}
\setCJKsansfont{HiraginoSansGB-W6}
\setCJKmonofont{FZLTXHK--GBK1-0}
\CTEXsetup[format+={\raggedright}]{section}

\usepackage{fontspec}
\setmainfont{TimesNewRomanPSMT}
\setsansfont{Verdana}
\setmonofont{CourierNewPSMT}

\usepackage{float}

\usepackage{geometry}
\geometry{a4paper,left=3cm,right=3cm,top=2.5cm,bottom=2.5cm}

\usepackage{fancyhdr}
\pagestyle{fancy}
\fancyhf{}
\chead{\itshape 现代密码学简介}
\rhead{\thepage}
\renewcommand\headrulewidth{0.6pt}

\makeatletter
\renewcommand\chapter{\if@openright\cleardoublepage\else\clearpage\fi
    \thispagestyle{fancy}% original style: plain
    \global\@topnum\z@
    \@afterindentfalse
    \secdef\@chapter\@schapter%
}
\makeatother

\renewcommand{\labelitemii}{$\circ$}

\usepackage{amsmath}
\usepackage{esvect}
\usepackage{bm}
\usepackage{upgreek}
\def \paral {/\!/}
\usepackage{mathtools}
\usepackage{amssymb}
\usepackage{extarrows}
\usepackage{mathrsfs}
\usepackage[amssymb]{SIunits}
\renewcommand{\epsilon}{\varepsilon}
\renewcommand{\phi}{\varphi}
\newcommand{\ext}{\displaystyle}
\newcommand{\arsinh}{\mathrm{arsinh}\,}
\newcommand{\arcosh}{\mathrm{arcosh}\,}
\newcommand{\artanh}{\mathrm{artanh}\,}
\newcommand{\dif}{\mathop{}\!{}\mathrm{d}}
\newcommand{\ic}{\mathrm{i}}
\newcommand{\Tr}{\mathrm{T}}
\newcommand{\ke}{\mathrm{k}}
\newcommand{\p}{\mathrm{p}}
\newcommand{\Z}{\mathbb{Z}}

\def\mathbm#1{{\bm #1}}
\def\vec#1{\vv{\bm #1}}
\def\xvec#1#2{\vv{{\bm #1}_{#2}}}
\def\yvec#1#2{\vv{{\bm #1}^{#2}}}
\def\abs#1{\left| #1 \right|}
\def\pth#1{\left( {#1}\right)}
\def\brace#1{\left\{ {#1} \right\}}
\def\E#1#2{{\mathrm{E}_{#1}\left({#2}\right)}}
\def\D#1#2{{\mathrm{D}_{#1}\left({#2}\right)}}
\renewcommand{\rem}{{\bfseries Remark}}

\usepackage{hyperref}
\hypersetup{pdfauthor={张曙},
pdftitle={现代密码学简介},
hidelinks
}
\begin{document}
\tableofcontents
\thispagestyle{fancy}
\chapter{绪论}
\section{什么是密码学}
如果我们要求一个从未接触过密码学的人处理一段文字,把这段文字尽可能地加密,让别人无法破解。那么,大多数人在深思熟虑之后,总能提出一些加密算法。这时候,一般人的思路可能会有几个方向。\par
有的人的方向是把这段文字中的字母之间通过各种复杂的运算进行组合。比如说,把要加密的文字中每两个字母在字母表中的位置进行相加,形成密码。比如说,``cryptography''中,``cr''变成了$3+18=21$, ``yp''变成了$25+16=41$, ``cryptography''对应的密码就是``214135252133''。但这样的密码,显然忽略了一点:密码的可解密性。我们之所以要进行加密,是为了安全地传递信息。但接收信息者必须要有解密的方法。但这种加密方式则没有对应的解密方法。所以说,这不是一个合格的加密方式。\par
这时,足够聪明的人,则考虑到了解密的方法。他们想出了一些类似于古代使用的加密方法。比如说,著名的凯撒密码:把一段由英文字母组成的语句,每个字母都在字母表中往后移3个位置。``veni vidi vici''也就变成了``yhql ylgl ylfl''. 这种密码的解密方法,也就是每个字母在字母表中向前移动3个位置。当问起这种密码体制里最重要的是什么,大部分人都会说是加密算法。如果把加密算法告诉了别人,那这种密码就相当于被破解了。\par
如果进而问起一个密码体制的组成,大部分人的回答,就是加密算法。也许有些自诩比普通人聪明一点的人,还会加上一个解密算法,也就是,将处理过后的密码变为正常文字的算法。如果用$m$代表要被加密的文字,$c$代表被加密出来的密码,$\mathrm{E}()$代表加密算法,$\mathrm{D}()$代表解密算法,那么大多数人理解的密码学,本质就是以下这两个式子:
\begin{gather*}
    c=\mathrm{E}\pth{m}\\
    m=\mathrm{D}\pth{c}
\end{gather*}

\subsection{密钥}
但是,我们考虑一下实际的情况。根据我们之前的常识,会发现,如果要用密码来传递信息,首先通信的双方必须要在一个安全信道中传递一些额外的信息。比如说,告诉接收者加密的方式,或者更直接地,告诉接收者解密算法。不存在一种加密的方式,能不事先通过安全信道传递信息,而使得只有接收者能够解密。那么,为了使通信更加安全,通信双方对安全信道的使用应该尽可能少。如果Alice和Bob使用凯撒密码进行通信,而Alice事先在安全信道中要对Bob说“我的这种加密方式是把一段由英文字母组成的语句,每个字母都在字母表中往后移3个位置。”这段话是如此之长,而如果考虑是在战场上,由通信员用摩尔斯电码发送,那么花费的这么长的时间显然是不合理的。再者,现代的通信方式一般都是把信息编码成二进制进行传递,那么,相比把一段话编码成二进制,不如找一些数字来代表这个密码。在我们的这段思想博弈中,一个重要的概念呼之欲出。\par
拿凯撒密码为例,在其加密算法中,还有一个关键的量:3. 试想,如果通信双方Alice和Bob, Alice在场的所有人她用的是凯撒密码的加密方式加密她的话,又在安全信道中告诉Bob``4'', 意味着她使用的凯撒密码里,是将每个字母在字母表的位置上向后移动4个位置。那么,即使在场有人能偷听到Alice给Bob的密码,也无法破解Alice想说的是什么,只有掌握了规则“向后移动4个位置”的Bob, 才能正确地破解密码。\par
如果还是按照之前的说法,一个密码体制包括加密算法和解密算法,那么向后移动3个位置的凯撒密码和向后移动4个位置的凯撒密码,就成了两个密码体制。但是,这两种加密方式是极其类似的,差别也就只在于一个向后移动3个位置,一个向后移动4个位置。我们对于这些密码的研究,也许也就十分类似。因此,把这两种加密方式归类为同一种密码体制,似乎是更好的选择。因此,\textbf{密钥}就应运而生了。所谓密钥,我们可以粗略地理解成加密算法、解密算法中的参数,也就是我们之前说的3、4. 通过一个密码体制通信的双方,需要首先在保密信道中确定密钥。而一个密码体制,也就可以由加密算法、解密算法和密钥构成。因此,如果以$k$代表密钥,那么之前的式子就变成了
\begin{gather*}
    c=\E{k}{m}\\
    m=\D{k}{c}
\end{gather*}

这也就是现在通用的密码学。\par
那么我们思考几个问题:在一次加密通信过程中,加密算法和解密算法中使用的密钥是否必须要相同?密钥是否只能是一个数?可不可以没有密钥?\par
针对第一个问题,确实存在某些高端的技巧,使得加密算法和解密算法使用的密钥是不同的。事实上,在密码学领域中,根据加密算法和解密算法使用的密钥是否相同,人们将加密方式分为对称加密与非对称加密,分别对应使用同一密钥和使用不同密钥。非对称加密方式也许难以理解,我们也会在充分学习对称加密后再介绍非对称加密。所以,我们接下来讨论的对称加密过程中,请大家记住,加密算法和解密算法使用的是相同的密钥。\par
此外,对于对称加密算法,密钥也并不一定是一个数。比如说,加密算法
\[\E{a, b}{m}=am+b\]
其密钥为$\pth{a, b}$, 但我们仍认为其使用单一密钥,也就是说,把$\pth{a, b}$看作一个密钥。\par
此外,可不可能不存在密钥呢?确实有这样的密码体制,但这样却也十分不安全。比如说,加密算法
\[\mathrm{E}\pth{m}=m\]
就是一个无密钥的加密算法。
\section{密码体制的基本要素}
根据之前的讨论,我们就可以得出一个密码体制的基本要素:
\begin{itemize}
    \item 明文空间$\mathcal{M}$\par
    所有可以被加密算法加密的元素组成的集合,加密算法的定义域。明文空间的元素叫做明文$m\in\mathcal{M}$.\par
    例如,在凯撒密码中,明文空间就为所有由英文字母组成的字符串。
    \item 密文空间$\mathcal{C}$\par
    所有可以由加密算法输出的元素组成的集合,加密算法的值域。密文空间的元素叫做密文$c\in\mathcal{C}$.\par
    我们需要注意到的是,这里的值域,可以理解成二元函数$\mathrm{E}\pth{k, m}$的值域。比如说,对于加密算法
    \[\E{k}{m}=m^2+k^2\]
    其密文空间为$[0, +\infty)$而非$[k^2, +\infty)$.
    \item 密钥空间$\mathcal{K}$\par
    所有密钥组成的集合。密钥空间的元素叫做密钥$k\in\mathcal{K}$.\par
    在非对称加密中,密钥空间分为加密密钥空间和解密密钥空间。
    \item 加密算法$\E{k}{m}$\par
    根据密钥生成的特定算法,将明文转化为密文。
    \item 解密算法$\D{k}{c}$\par
    根据密钥生成的特定算法,将密文转化为明文。\par
    对于对称加密算法,也就是只使用一个密钥的加密体制,解密算法与加密算法满足
    \[\D{k}{\E{k}{m}}=m\]
\end{itemize}

在我们讨论密码体制的一些性质时,密钥生成算法有时也是必要的。什么是密钥生成算法呢?回忆之前Alice和Bob的例子,凭什么Alice选择的密钥是4而不是25呢?这就涉及到了密钥生成算法。在这个例子中,密钥生成算法就是Alice自己想到哪个密钥就输出哪个密钥。但是,从严格意义上来讲,密钥生成算法是一种概率算法。所谓概率算法,就是在算法的步骤中涉及到了某些概率。比如说,在某个密钥生成算法中,在$(0, 1)$中等概率随机生成一个数$t$, 而生成的密钥$k$满足
\[
    k=\begin{cases}1&t\in\pth{0.5, 1}\\0&t=0.5\\-1&t\in\pth{0, 0.5}\end{cases}
\]

这就是一个典型的概率算法。其特点就是在两次运行中输出的结果不一定相同。\par
因此,我们称一个加密方案包含三个要素:加密算法$\mathrm{E}$, 解密算法$\mathrm{D}$, 密钥生成算法$\mathrm{G}$. 根据定义,我们可以说,一个密码体制由一个加密方案$\pth{\mathrm{E}, \mathrm{D}, \mathrm{G}}$及一个明文空间$\mathcal{M}$完全定义。\par
因此,Alice和Bob的一次加密通信的过程包括:
\begin{enumerate}
    \item Alice根据密钥生成算法$\mathrm{G}$生成密钥$k\in\mathcal{K}$.
    \item Alice通过安全信道将$k$告诉Bob.
    \item Alice将想要传达的明文$m\in\mathcal{M}$根据加密算法加密成密文$c=\E{k}{m}$告诉Bob.
    \item Bob根据之前Alice告诉自己的$k$、密文$c$及解密算法得出明文$m=\D{k}{c}$
\end{enumerate}

那么,在一个密码体制中,哪个最重要呢?是不是之前我们说的加密算法呢?这里,就不得不提Kerckhoffs原则。用现代的语言来说,Kerckhoffs原则阐述的是:
\begin{quote}
    提倡安全性不能建立在对算法的保密上。
\end{quote}

也就是说,我们如果要证明一个加密体制的安全性,不能指望算法的保密性。我们应默认加解密算法可以被所有人知道(事实上也确实如此)。也就是说,真正值得保密的,是密钥。如果潜在的敌手获得密钥,那么根据公开的解密算法,那么他就可以从窃得的密文中获得明文。
\section{加密系统的安全性}
为了从数学上定义加密系统的安全性,我们必须引入一些和概率有关的定义。
\end{document}